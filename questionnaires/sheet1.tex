\documentclass[12pt,a4paper]{article}
\usepackage[margin=1in]{geometry}
\usepackage{listings}
\usepackage{xcolor}
\usepackage{setspace}

\setlength{\parindent}{0pt}

\lstset{
    language=C,
    basicstyle=\ttfamily\small,
    frame=single,
    backgroundcolor=\color{gray!5},
    showstringspaces=false,
    columns=fullflexible
}

\begin{document}
\begin{center}
{\LARGE \bf C Pointer Questionnaire\par}
\end{center}

\vspace{1em}

\noindent
\textbf{Roll Number:} \underline{\hspace{5cm}} 
\hfill
\textbf{Time Allotted: 30 minutes}

\vspace{1em}
\hrule
\vspace{1em}

\textit{\textbf{Instructions:}} \\
Each question consists of a snippet of code in the \textit{C} programming language. Your task is to create a clear representation of the code in an appropriate format (e.g., figures, text, etc.) in the space provided and use it to evaluate the output of the code.

\section*{Question 1}
\begin{lstlisting}
int x = 10, y = 20;
int *p = &x;
int *q = &y;
*p = *q;
q = p;
*q = 50;
printf("%d %d\n", x, y);
\end{lstlisting}

\vspace{36em}

\section*{Question 2}
\begin{lstlisting}
int arr[] = {2, 4, 6, 8};
int *p = arr;
printf("%d\n", *(p + 2));
\end{lstlisting}

\vspace{50em}

\section*{Question 3}
\begin{lstlisting}
char str1[] = "Hi";
char str2[] = "Bye";
char *p = str1;
char *q = str2;
*p = *q;
str2[1] = 'a';
printf("%s %s\n", str1, str2);
\end{lstlisting}

\end{document}
