\documentclass[12pt,a4paper]{article}
\usepackage[margin=1in]{geometry}
\usepackage{listings}
\usepackage{xcolor}
\usepackage{setspace}

\setlength{\parindent}{0pt}

\lstset{
    language=C,
    basicstyle=\ttfamily\small,
    frame=single,
    backgroundcolor=\color{gray!5},
    showstringspaces=false,
    columns=fullflexible
}

\begin{document}
\begin{center}
{\LARGE \bf C Pointer Questionnaire\par}
\end{center}

\vspace{1em}

\noindent
\textbf{Roll Number:} \underline{\hspace{5cm}} 
\hfill
\textbf{Time Allotted: 30 minutes}

\vspace{1em}
\hrule
\vspace{1em}

\textit{\textbf{Instructions:}} \\
Each question consists of a snippet of code in the \textit{C} programming language. Your task is to create a clear representation of the code in an appropriate format (e.g., figures, text, etc.) in the space provided and use it to evaluate the output of the code.

\section*{Question 1}
\begin{lstlisting}
int s = 3;
int *temp = &s;
int **p = &temp;
int *q = *p;
*q = 10;
printf("%d %d %d\n", s, *temp, **p);
\end{lstlisting}

\vspace{33em}

\section*{Question 2}
\begin{lstlisting}
int arr[] = {1, 2, 3, 4, 5};
int *p = arr;
*(p + 2) = *(p + 4);
p++;
printf("%d %d\n", *p, *(p + 2));
\end{lstlisting}

\vspace{48em}

\section*{Question 3}
\begin{lstlisting}
char str1[] = "Hello";
char str2[] = "World";
char *p = str1;
char *q = str2;
p = q;
*str2 = 'h';
*(q+2) = 'v';
printf("%s %s\n", p, str2);
\end{lstlisting}

\vspace{18em}

\end{document}
